\section{Organizacja} 
\subsection{Sponsor}
Sponsorem będzie sam prezydent Islandii.
Będzie on zarządzał finansowaniem projektu, oraz wyznaczał ogólne wytyczne.

Będzie bezpośrednio brał udział w rozmowach z największymi sponsorami, wyznaczał kształt zarządu, oraz sprawował ogólną władzę nad wszystkim.
Każda jego decyzja będzie jawna publicznie.

\subsection{Komitet sterujący}
Są to naukowcy powołani przez polityków. Często będąc ich wcześniejszymi pomocnikami.
W skład wchodzą osoby z największych islandzkich zakładów, a także zaproszeni znani naukowcy z poza Islandii.

Ich zadaniem jest ocena, czy projekt idzie w dobrym kierunku i podejmowanie najlepszych decyzji.

\subsection{Kontrolerzy}
Wyznaczeni przez komitet sterujący odpowiadają za rozwiązywanie małych problemów między zespołami.
Zapewniają jakość jednocześnie ją kontrolując.

\subsection{Kierownik gałęzi}
Kierownicy są powoływani przez prezydenta, każdy specjalizuje się w innej dziedzinie i zarządza znanymi dla siebie zespołami.
Korzystając z pomocy komitetu sterującego każdy z nich podejmuje mniej znaczące decyzje.

Wraz z odpowiednimi osobami z komitetu będzie chodził po budowanych strukturach i na bieżąco monitorował jakość i wykonanie prac.
Będzie wymagał pokazania, czy budowane struktury działają, oraz zbierał wszystkie informacje o postępie.

\subsection{Główny projektant}
Jest przedstawicielem komitetu sterującego. Nie może sam projektować całego systemu, dlatego osoba na tym stanowisku zmienia się w zależności od dziedziny jaką kontroluje.

Wraz z kierownikiem projektu uczestniczy w kontroli odpowiednich dla siebie części systemu.

\subsection{Pełnomocnik do spraw jakości}
Jest powołany przez prezydenta. Kontroluje, czy ogólnie proces przebiega poprawnie.
Odpytuje kierowników o ewentualne opóźnienia i sumuje wszystkich razem.

\subsection{Zespoły}
Każdy zespół składa się z wyselekcjonowanych przez komitet sterujący osób.
Niektóre zespoły przyszły razem, gdy na przykład wcześniej razem pracowały w innej firmie.
Wiele zespołów zostało zgłoszonych przez sponsorów.

Każdy zespół zajmuje się małym fragmentem projektu. Na przykład jeden zespół może zająć się budową odcinka toru dla wewnętrznego transportu, inny zwrotnicami.
Jeden zespół będzie ustawiał bramki do sprawdzania pojazdów na nadajniku, a inny projektował moduł sterowników do obsługi jednego z kryształów.

Lider zespołu musi dobrze znać się na temacie, aby wiedzieć jak postępują prace i jakie są problemy. Będzie on raportował do i przyjmował wizyty swojego kierownika projektu.

\subsection{Zasoby materialne}
W początkowych fazach projektu powstanie miasteczko wokół całej struktury. Będą to hotele, centra rozrywki, biura, ujęcia wody, struktura prądowa itp.
Pracownicy, którzy będą budować te struktury początkowo zamieszkają w barakach z kontenerów. Potem w miarę rozwoju będą mogli się przenosić od najtańszych budynków na hotele do coraz wygodniejszych miejscówek.
Także do bardziej podstawowych prac nie potrzeba wyższych wymagań, jak szybki internet.

Wtedy też dołączą bardziej wykwalifikowani pracownicy, którzy mieszkając w hotelach i pracując w biurach będą zajmować się innymi częściami projektu.

Oczywiście hotele docelowo nie służą pracownikom na stałe, będą mieszkać w połowicznie skończonych budynkach, które ciągle będą dokańczane, a biura po skończonej budowie zostaną sprzedane innym firmom.

Początkowo większość zarządu będzie wygodnie siedzieć w Reykjaviku wysyłając często pojedyncze osoby do kontroli prac budowlanych, potem wszyscy powinni przenieść się na miejsce, aby dokładnie kontrolować postęp.

Infrastruktura telekomunikacyjna powinna być zbudowana jako jedna z pierwszych, bo to na niej bazują pracownicy.

\subsection{Centrala}
Główne biuro projektu z którego wysyła się osoby i komunikaty.
To tam mogą się zgłaszać liderzy zespołów w razie problemów i tam znajdują się sale spotkań.

W najbliższej okolicy centrali powinny mieszkać osoby z zarządu, aby szybko móc do niej dojść.

Każdy z zarządu ma określone dyżury w centrali w których musi być do dyspozycji po uprzednim umówieniu się na spotkanie.
Powinien móc zorganizować spotkanie jeszcze tego samego dnia, albo następnego, aby uniknąć niepotrzebnych przestojów w pracy.