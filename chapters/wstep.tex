\section{Wstęp}
\subsection{Opis}
Dnia 5 stycznia 2031 roku Ziemia otrzymała wiadomość o chęci zawiązania współpracy z kosmicznym gatunkiem Handthów.
Wiadomość była jasna i klarowna, skierowana do obywateli wszystkich krajów Ziemi.
Proponowano nam podłączenie do ogólnogalaktycznej sieci urządzeń teleportujących, aby umożliwić łatwy i szybki handel z 24 innymi cywilizacjami, niektórymi nawet mniej rozwiniętymi od Ziemi.
Jeśli się zgodzimy, Handthowie wylądują we wskazanym przez nas miejscu i dostarczą wymagane materiały do budowy odbiornika i nadajnika.

Zależało im na klarowności rozmów i dowolności naszego rozwiązania tak, aby nie zostać posądzonym o chęć zawładnięcia Ziemią.
Ważne było, abyśmy to my mieli kontrolę nad naszym urządzeniem, tak jak państwa mają kontrolę nad swoimi przejściami granicznymi.
Oczywiście pojawiło się bardzo wiele protestów na temat przystępowania do czegoś takiego, jednak wbrew wszystkim filmom o kosmitach tym razem wszystkie kraje zostały potraktowane po równo.
Debata polityczna o tym, czy największe mocarstwa mogą decydować o terenach położonych poza swoimi granicami została przerwana przez Islandię, której to obywatele w 90\% zgodzili się na budowę urządzenia na swoim terenie.

\subsection{Realizacja}
Rząd Islandii zorganizował zbiórkę pieniędzy na budowę, do której dołączyło wiele międzynarodowych korporacji.
W zamian za to mieli mieć zniżki na korzystanie z infrastruktury, a także olbrzymią reklamę.
Niezależne grupy inżynierów zgłosiły się do pomocy przy budowie, niektórzy nawet nie oczekiwali niczego w zamian (może chcieli potem wyemigrować z Ziemi).

Milcząco założono, że ponieważ przedsięwzięcie będzie się odbywać na terenie Islandii, to jej politycy i obywatele będą decydować o jej kształcie.
Wyłoniono reprezentację naukowców w ogólnym głosowaniu, która z pomocą wybranych osób będzie stanowić zarząd.
Najwięksi sponsorzy będą mieli bezpośredni wpływ na zarząd i będą aktywnie uczestniczyć w podejmowaniu decyzji.
Ich głosy powinny być proporcjonalne do wkładu.

Teleporter będzie położony w Północno-Wschodniej części wyspy, kilkadziesiąt kilometrów od wielkiej elektrowni geotermalnej.
Ten trudny i rozłożysty teren pozwoli na szeroką, a zarazem tanią budowę, a jednocześnie będzie tworzył naturalną barierę przed ewentualnymi atakami wojsk wrogich krajów.
Całość ma za zadanie służyć wiele wieków, zatem musi być odporna na trzęsienia ziemi, zmiany klimatu i katastrofy.

Każda grupa będzie zajmować się jednym aspektem budowy przedsięwzięcia.
Od zespołów mechaników i logistyków projektujących wewnętrzny transport do informatyków piszących sterowniki do anten i kontroli głównych modułów.
