\chapter{Wstęp}
\section{Opis}
Dnia 5 stycznia 2031 roku Ziemia otrzymała wiadomość o chęci zawiązania współpracy z kosmicznym gatunkiem Handthów.
Wiadomość była jasna i klarowna, skierowana do obywateli wszystkich krajów Ziemi.
Proponowano nam podłączenie do ogólnogalaktycznej sieci urządzeń teleportujących, aby umożliwić łatwy i szybki handel z 24 innymi cywilizacjami, niektórymi nawet mniej rozwiniętymi od Ziemi.
Jeśli się zgodzimy, Handthowie wylądują we wskazanym przez nas miejscu i dostarczą wymagane materiały do budowy odbiornika i nadajnika.

Zależało im na klarowności rozmów i dowolności naszego rozwiązania tak, aby nie zostać posądzonym o chęć zawładnięcia Ziemią.
Ważne było, abyśmy to my mieli kontrolę nad naszym urządzeniem, tak jak państwa mają kontrolę nad swoimi przejściami granicznymi.
Oczywiście pojawiło się bardzo wiele protestów na temat przystępowania do czegoś takiego, jednak wbrew wszystkim filmom o kosmitach tym razem wszystkie kraje zostały potraktowane po równo.
Debata polityczna o tym, czy największe mocarstwa mogą decydować o terenach położonych poza swoimi granicami została przerwana przez Islandię, której to obywatele w 90\% zgodzili się na budowę urządzenia na swoim terenie.

\section{Realizacja}
Powołano zbiórkę pieniędzy na budowę, do której dołączyło wiele międzynarodowych korporacji.
W zamian za to mieli mieć zniżki na korzystanie z infrastruktury.
Niezależne grupy inżynierów zgłosiły się do pomocy przy budowie, niektórzy nawet nie oczekiwali niczego w zamian (może chcieli potem wyemigrować z Ziemi).

Teleporter będzie położony w Północno-Wschodniej części wyspy, kilkadziesiąt kilometrów od wielkiej elektrowni geotermalnej.
Ten trudny i rozłożysty teren pozwoli na szeroką, a zarazem tanią budowę, a jednocześnie będzie tworzył naturalną barierę przed ewentualnymi atakami wojsk wrogich krajów.
Całość ma za zadanie służyć wiele wieków, zatem musi być odporna na trzęsienia ziemi, zmiany klimatu i ataki.

Nadajnik i odbiornik to dwie oddzielne infrastruktury, bardzo od siebie różne. Nasz zespół będzie miał za zadanie oprogramować jedynie nadajnik.
Jest on bardziej skomplikowany od odbiornika, ale mniej podatny na błędy, przez co łatwiejszy w testowaniu.

\section{Technologie}
Otrzymaliśmy od Handthów dwa wielkie kryształy, jeden pozwalał na zamianę materii w energię, a drugi odwrotnie. 
Stanowić one będą rdzeń całego urządzenia. 
Dołączone do niego są instrukcje obsługi, a także plany budowy.
Postanowiono na ich podstawie zbudować ziemskie kopie, a oryginały trzymać jako zapasowe, aby mieć pełną kontrolę nad ich działaniem.

Interakcja polega na wysyłaniu sygnałów elektrycznych, aby sterować ich dokładną pracą.
Można ustawić oczekiwane napięcie na wejściu i wyjściu.
Kryształ wysyła promień w kierunku swojego wierzchołka, który zamienia znalezioną energię w pakiety energii.
Drugą stroną wysyła energię do przekaźnika w kosmosie, który to kieruje do następnego itd. aż do odbiornika.
Budową i programowaniem anteny, która prześle i skieruje tą energię zajmie się inny zespół.

Sterujące sygnały elektryczne są binarne bez podziału na bajty.
Można użyć standardowych wyjść kontrolerów.
Ilość wejść i wyjść sięga kilku tysięcy.

Zdecydowano się nie używać rozwiązań opartych o systemy operacyjne, a rozwiązać to za pomocą kontrolerów czasu rzeczywistego.
\begin{itemize}
\item Do budowy użyje się nowych procesorów otwartej architektury HUS autorstwa Politechniki Warszawskiej.
\item Programowanie takich układów będzie w C z elementami Asemblera. 
\item Kompilacja za pomocą \texttt{gcc-hus}.
\item Testy jednostkowe \texttt{boost}.
\end{itemize}
