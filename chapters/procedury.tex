\section{Procedury}
Należy zdefiniować wszystkie wymagane procedury, aby stworzyć zbiór zasad wedle którego wszystkie osoby mają postępować.
To pozwoli przyspieszyć działanie i uniknie niejasności.
\subsection{Komunikacja}
Komunikacja jest podstawą działania, musi być jasna i szybka. Faworyzuje się proste i nieformalne wypowiedzi, aby oszczędzać czas i nie powodować stresu.
Zamiast spędzać wiele godzin na pisaniu składniowo poprawnego listu, pracownicy powinni przekazać informację w kilku słowach.

Będą używać do tego głównego systemu, wypełniając formularze, które upraszczają przekazywanie informacji.
\subsubsection{Komunikacja wewnętrzna}
Każdy pracownik, zarówno fizyczny jak i umysłowy, będzie posiadał skrzynkę pocztową, za pomocą której komunikuje się z dowolną inną osobą.
Dodatkowo wyposażany będzie w swój własny klucz GPG i będzie posiadał publiczne klucze innych osób.
To pozwoli na bezpieczną prywatną komunikację między pracownikami.

Pomimo, że jedną z głównych cech projektu jest transparentność, istnieją tematy, które dla względów bezpieczeństwa muszą być omawiane prywatnie.

Liderzy zespołów zgłaszać będą swoje problemy do kierownika projektu, a ten przekaże maila dalej do komitetu sterującego, jeśli problem jest poważny.

Jednak lepiej jest, jeśli przedstawiciel osobiście uda się do centrali porozmawiać z kierownikiem, lub komitetem sterującym.
Powinni oni dać mu wskazówki i pomóc rozwiązać problemy.

\subsubsection{Komunikacja zewnętrzna}
Prezydent i sponsorzy będą bardzo mocno uczestniczyć w projekcie. Wskazane jest, aby mogli oglądać postępy prac dla odpowiednich dla siebie dziedzin.
Komunikacja powinna być szybka i prosta o takich samych zasadach, jak wewnętrzna.

\subsection{Zapewnienie jakości}
Ogólną jakość zapewnią sami pracownicy poprzez fakt, że ich lider w kązdej chwili będzie mógł skomunikować się z projektantem i naukowcami.
Dzięki temu problemy będą mogły być natychmiastowo rozwiązane.

Główny serwer będzie monitorował i wyświetlał informacje nt. wszystkich zespołów.
Dzięki temu widać będzie, jak w danej chwili radzi sobie każdy zespół.

\subsection{Kontrola jakości}
Liderzy zespołów zgłaszać się będą do okresowej kontroli w dogodnym dla siebie terminie lub termin będzie im przydzielany z góry.

W skłąd komisji kontrolującej wejdą: kierownik gałęzi tematycznie powiązanej z pracą zespołu, projektant kontrolowanego systemu i oficer jakości znający tą dziedzinę.
Do pomocy przy skomplikowanych pracach będzie mógł do nich dołączyć także wybrany naukowiec.

W przypadku kontroli części projektu, za którą odpowiadają różne zespoły z różnych dziedzin, dwie komisje będą mogły się połączyć i razem kontrolować część.

\subsection{Kontrola postępów}
Główny serwer będzie miał zadane bramki czasowe, które każdy zespół ma spełnić w określonym czasie. Na bieżąco widać będzie, kto się spóźnia i dlaczego, oraz jaki procent budżetu został wykorzystany.
Co więcej, informacje o opóźnieniach będą dostępne publicznie w celu ułatwenia uzyskania pomocy od innych zespołów przez zespoły, w których wystąpił problem.

\subsection{Kontrola zmian}
Zmiany będą inicjowane przez rozmowę lidera zespołu z projektantem po tym, gdy dojdą do wniosku, że nie da się wykonać zadanej pracy nie przekraczając czasu albo finansów.
O zmianie informowany będzie także kierownik gałęzi.

W zależności od powagi zmiany, do sprawy zaprzęgnięty może zostać także główny projektant, koordynator lub komitet sterujący.
W przypadku skomplikowanych kwestii technicznych o pomoc powinno się poprosić także naukowców.

Informacja o zmianie zostanie ogłoszona w systemie komputerowym w ramach specjalnego rejestru zmian, a także miala informacyjnego do wszystkich pracowników, na których pracę zmiana może meić wpływ.

\subsection{Kontrola problemów}
Jeśli problem będzie interdyscyplinarny, liderzy zespołów powinni się razem spotkać w celu rozwiązania sporu.
Jeśli to nie wystarczy, możliwa może być pomoc naukowców albo projektanta systemu.

Problemy jednej grupy będą zazwyczaj rozwiązywane najpierw poprzez pomoc naukowców, a potem projektanta systemowego.


