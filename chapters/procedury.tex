\section{Procedury}
Należy zdefiniować wszystkie wymagane procedury, aby stworzyć zbiór zasad wedle którego wszystkie osoby mają postępować.
To pozwoli przyspieszyć działanie i uniknie niejasności.
\subsection{Komunikacja}
Komunikacja jest podstawą działania, musi być jasna i szybka. Faworyzuje się proste i nieformalne wypowiedzi, aby oszczędzać czas i nie powodować stresu.
Zamiast spędzać godzin na pisaniu składniowo poprawnego listu, pracownicy powinni przekazać informację w kilku słowach, gdyż mają dużą ilość innych maili.

Będą używać do tego głównego systemu wypełniając formularze, które upraszczają przekazywanie informacji.
\subsubsection{Komunikacja wewnętrzna}
Każdy pracownik zarówno fizyczny i psychiczny ma daną skrzynkę pocztową za pomocą której komunikuje się z dowolną inną osobą.
Dodatkowo wyposażany jest w swój własny klucz GPG i posiada publiczne klucze innych osób.
To pozwala na bezpieczną prywatną komunikację między osobami.

Pomimo, że jedną z głównych cech projektu jest transparentność, to istnieją tematy, które dla względów bezpieczeństwa muszą być omawiane prywatnie.
Daje to także możliwość większego otwarcia się do drugiej osoby tak samo, jak gdyby rozmawiali prywatnie.

Liderzy zespołów zgłaszają swoje problemy do kierownika projektu, a ten przekazuje maila dalej do komitetu sterującego, jeśli problem jest poważny.

Jednak lepiej jest, jeśli przedstawiciel osobiście uda się do centrali porozmawiać z kierownikiem, lub komitetem sterującym.
Powinni oni dać mu wskazówki i rozwiązywać problemy.

\subsubsection{Komunikacja zewnętrzna}
Prezydent i sponsorzy bardzo mocno uczestniczą w projekcie. Wskazane jest, aby mogli oglądać postępy prac dla odpowiednich dla siebie dziedzin.
Komunikacja powinna być szybka i prosta o takich samych zasadach, jak wewnętrzna.

\subsection{Zapewnienie jakości}
Ogólną jakość zapewniają sami pracownicy poprzez fakt, że ich lider zawsze może skomunikować się z projektantem i naukowcami.
Dzięki temu problemy mogą być natychmiastowo rozwiązane.

Główny serwer monitoruje i wyświetla informacje wszystkich zespołów.
Dzięki temu widać jak w danej chwili radzi sobie każdy zespół.

\subsection{Kontrola jakości}
Liderzy zespołów zgłaszają się do okresowej kontroli w dogodnym dla siebie terminie, lub termin jest im przydzielany z góry.

Grupa kontrolna w której skład wchodzą kierownik gałęzi tematycznie powiązanej z pracą zespołu, projektant kontrolowanego systemu i oficer jakości znający tą dziedzinę.
Do pomocy przy skomplikowanych pracach może do nich dołączyć także jakiś naukowiec.

W przypadku kontroli części projektu za którą odpowiadają różne zespoły różnych dziedzin, dwie grupy kontrolne mogą się połączyć i razem kontrolować część.

\subsection{Kontrola postępów}
Główny serwer ma zadane bramki, które każdy zespół ma spełnić w określonym czasie. Na bieżąco widać, kto się spóźnia i dlaczego, oraz jaki budżet wykorzystał z zadanego.
Co więcej każdy może sprawdzić dlaczego tak się dzieje i pomagać tym zespołom bardziej, niż innym.

\subsection{Kontrola zmian}
Zmiany zazwyczaj są inicjowane przez rozmowę lidera zespołu z projektantem po tym, gdy doją do wniosku, że nie da się wykonać zadanej pracy nie przekraczając czasu, albo finansów.
O zmianie informowany jest także kierownik gałęzi.

W zależności od powagi zmiany do sprawy zaprzęgnięty może zostać także główny projektant, koordynator, lub komitet sterujący.
Do pomocy powinno się użyć którychś z naukowców.

Informacja o zmianie zostaje ogłoszona w systemie komputerowym.

\subsection{Kontrola problemów}
Jeśli problem dotyczy interdyscyplinarności, liderzy zespołów powinni się razem spotkać w celu rozwiązania sporu.
Jeśli to nie wystarczy możliwa może być pomoc naukowców, albo projektanta systemu.

Problemy jednej grupy są zazwyczaj rozwiązywane najpierw poprzez pomoc naukowców, a potem projektanta systemowego.


