\section{Procedury}
Należy zdefiniować wszystkie wymagane procedury, aby stworzyć zbiór zasad wedle którego wszystkie osoby mają postępować.
To pozwoli przyspieszyć działanie i uniknie niejasności
\subsection{Komunikacja}
Komunikacja jest podstawą działania, musi być jasna i szybka. Faworyzuje się proste i nieformalne wypowiedzi, aby oszczędzać czas i nie powodować stresu.
Zamiast spędzać godzin na pisaniu składniowo poprawnego listu, pracownicy powinni przekazać informację w kilku słowach, gdyż mają dużą ilość innych maili.
\subsubsection{Komunikacja wewnętrzna}
Każdy pracownik zarówno fizyczny i psychiczny ma daną skrzynkę pocztową za pomocą której komunikuje się z dowolną inną osobą.
Dodatkowo wyposażany jest w swój własny klucz GPG i posiada publiczne klucze innych osób.
To pozwala na bezpieczną prywatną komunikację między osobami.

Pomimo, że jedną z głównych cech projektu jest transparentność, to istnieją tematy, które dla względów bezpieczeństwa muszą być omawiane prywatnie.
Daje to także możliwość większego otwarcia się do drugiej osoby tak samo, jak gdyby rozmawiali prywatnie.

Liderzy zespołów zgłaszają swoje problemy do kierownika projektu, a ten przekazuje maila dalej do komitetu sterującego, jeśli problem jest poważny.

Jednak lepiej jest, jeśli przedstawiciel osobiście uda się do centrali porozmawiać z kierownikiem, lub komitetem sterującym.
Powinni oni dać mu wskazówki i rozwiązywać problemy.

\subsubsection{Komunikacja zewnętrzna}
Prezydent i sponsorzy bardzo mocno uczestniczą w projekcie. Wskazane jest, aby mogli oglądać postępy prac dla odpowiednich dla siebie dziedzin.
Komunikacja powinna być szybka i prosta o takich samych zasadach, jak wewnętrzna.

\subsection{Zapewnienie jakości}
Komitet sterujący odpowiada za wyznaczanie osób do kontroli jakości.
To naukowcy z komitetu przyjmują podania o prace i akceptują nowych pracowników.

Dodatkowo w centrali będzie stać serwer informacyjny na którym każdy będzie mógł sprawdzić postęp prac każdej z grup, komunikować się grupowo z innymi i robić ustalenia.
Będzie mógł dostać informację o pracownikach, gdzie może ich znaleźć itp.
Serwer będzie podsumowywał cały projekt.

Kontrolerzy mają za zadanie siedzieć przed serwerem i rozwiązywać zgłoszone przez zespoły problemy.

\subsection{Kontrola jakości}
Co jakiś czas kontrolerzy udają się na miejsce budowy, najlepiej wraz z kierownikami gałęzi, aby sprawdzić, czy informacje zgłoszone na serwerze mają pokrycie w rzeczywistości.
Kontrolują oni wraz z listą problemów, czy zostały one rozwiązane i w jaki sposób.

\subsection{Kontrola postępów}
Główny serwer ma zadane bramki, które każdy zespół ma spełnić w określonym czasie. Na bieżąco widać, kto się spóźnia i dlaczego, oraz jaki budżet wykorzystał z zadanego.
Co więcej każdy może sprawdzić dlaczego tak się dzieje i pomagać tym zespołom bardziej, niż innym.

\subsection{Kontrola zmian}
O każdej zmianie kontrolerzy informują komitet sterujący, który decyduje o powadze zmiany i jak ją wprowadzić.
Po akceptacji informacja jest umieszczana na serwerze, aby była dobrze widoczna.

Kontrolerzy później kontrolują, czy zespoły dobrze sobie radzą ze stosowaniem się do niej.

\subsection{Kontrola problemów}
Kontrolerzy zajmują się także tym. Każdy problem w zależności od wielkości jest rozwiązywany przez różną liczbę przedstawicieli komitetu sterującego.
System automatycznie dobiera ich w zależności od tematu w jakim się specjalizują, oraz tak, aby rozłożyć pracę po równo.

Decyzja jest ustawiana na serwerze, następnie kontrolerzy po pewnym czasie wracają sprawdzić, czy problem został rozwiązany.

O każdym problemie w zespole informowany jest także kierownik gałęzi.



