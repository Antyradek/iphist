\section{Analiza ryzyka}
Największe ryzyka, jakie grożą projektowi.
\subsection{Duża skala}
Duża skala przedsięwzięcia wymaga zgranej współpracy wielu zespołów. Tam zawsze powstaje najwięcej problemów. 

Dlatego ważna jest spójna komunikacja i przekazywanie informacji. System informatyczny oraz transport są jednymi z pierwszych struktur, które trzeba będzie zbudować.
Należy także mieć w gotowości standardowy transport na wypadek awarii obecnego.

\subsection{Nieznana technologia}
Technologia jest nieznana. Pomimo wielu udanych testów nadal istnieje ryzyko, że coś może nie zadziałać tak, jak trzeba. Nikt nie ma doświadczenia z obchodzeniem się z nią.

Najważniejsza część projektu to kryształy do teleportacji. Bez ich działających nie ma sensu budować całej otoczki.
Jeśli nie uda się stworzyć ich systemu, cały projekt okaże się całkowitą porażką.

Dlatego ta jedna rzecz powinna być stworzona jako pierwsza na swoim miejscu, korzystając z tymczasowych rozwiązań transportowych i komunikacyjnych.
Całość musi działać, nawet jeśli nie będzie miała podstawowych udogodnień i zapewnienia bezpieczeństwa.

Na szczęście zawsze można przerobić nieudany projekt na nowe miasto na pustkowiu.
Jeśli wprowadzić liberalne prawo, nawet zadeklarować jako osobne państewko, to miasto mogłoby się rozwijać pomimo, że pierwotnie miało być przeznaczone do czegoś innego.

\subsection{Mieszany zespół}
Zespół składa się z wielu osób z całego świata. Pomimo dużej kontroli i restrykcyjnych warunków, niektórzy mogą nie mieć dość profesjonalizmu, aby dobrze pracować. Powstają też bariery kulturowe i językowe. 

Należy umieszczać podobne osoby w jednym zespole, aby uniknąć wewnętrznych tarć. Mimo że mieszany zespół teoretycznie może być szerzej wykształcony, to wewnętrzny spokój mentalny jest ważniejszy.
Także komunikacja między zespołami powinna być prosta. Najlepiej bez pośredników i tłumaczy. Lider zespołu powinien móc dogadać się ze wszystkimi innymi.

\subsection{Zła kolejność budowy}
Zaczynanie budowy od zera wiąże się z odpowiednią kolejnością budowy systemów, jest duże ryzyko, że nie zawsze zapewni się wszystkie wymagania, co może spowolnić pracę innych. 

Lepiej jest wydłużyć czas budowy używając mniej zespołów na raz, a zapewnić większą kontrolę. Dzięki temu czasem można zapewnić coś tymczasowo nie wstrzymując rozwoju.
Jednakże takie akcje niepotrzebnie zużywają budżet, dlatego planowanie jest ważniejsze.

\subsection{Organizacja}
Organizacja tych wszystkich osób jest wyzwaniem, które bardzo trudno spełnić. Trzeba dowieść surowce, a także pożywienie dla wszystkich. 

Tak, jak poprzednim razem, zmniejszenie jednoczesnej liczby pracujących pomoże zapanować nad wszystkimi i stosować tymczasowe rozwiązania.
Wydłuża to czas pracy, jednakże limit nie jest sztywno określony.
 
\subsection{Modyfikacje planu}
Plan projektu nigdy nie będzie idealny, zwłaszcza jeśli ma obejmować swoim zasięgiem wszystko. Być może trzeba będzie modyfikować go w locie. 

Aby uniknąć tego, należy poświęcić na projektowanie więcej czasu. Należy myśleć w przód i dać do zweryfikowania wielu różnym osobom, aby ich inne myślenie pomogło wykryć słabe punkty



