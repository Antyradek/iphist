\section{Analiza ryzyka}
Największe ryzyka, jakie grożą projektowi od najgroźniejszego, najbardziej prawdopodobnego:
\begin{description}
 \item[Duża skala] Skala projektu jest na tyle wielka, że istnieje bardzo wiele elementów powodujących ryzyko upadku projektu.
 \item[Przekroczenie czasu budowy] Projekt zostanie ukończony poprawnie, ale w za długim czasie.
 \item[Przekroczenie budżetu] Może być spowodowane bardzo wieloma czynnikami. Jest najpopularniejszym zagrożeniem większości projektów.
 \item[Nieznana technologia] Nowe miasta bardzo rzadko buduje się od podstaw. Również budowa stacji teleportujących nigdy wcześniej nie była przeprowadzana.
 \item[Zła specyfikacja] Może powodować, że trzeba będzie wprowadzać kosztowne zmiany, działać na obiektach tymczasowych, albo powodować, że struktury będą niepoprawnie działać.
 \item[Zła kolejność budowy] Nawet jeśli system jest poprawnie zaprojektowany, trzeba jeszcze stworzyć plan budowy. Bez niego budowniczy wyższych systemów mogą nie mieć wymaganych niższych i infrastruktury do życia. Będzie to generować dodatkowe koszty.
 \item[Przeciągnięcie kosztów budowy modułu] Jest bardzo prawdopodobne, że koszt wykonania pracy przez jedną ekipę może być większy, niż zaplanowano na skutek błędów i zdarzeń niezależnych.
 \item[Przeciągnięcie czasu budowy modułu] Każdy zespół buduje coś, na czym kolejni będą bazować. Przedłużenie budowy jednej części pociąga za sobą opóźnienia wszystkich innych.
 \item[Brak kompetencji zespołów] Nawet jeśli ogólna infrastruktura postanie poprawnie, to najtrudniej jest zbudować najbardziej zaawansowane systemy obsługujące kryształy. Nikt wcześniej nie próbował robić takich rzeczy i nikt nie ma doświadczenia.
 \item[Niepoprawna logistyka] Jeśli transport towarów i osób nie będzie poprawnie stworzony, to projekt z pewnością będzie powodował opóźnienia. Mogą powstać braki w zaopatrzeniu nie tylko materiałów budowlanych, ale także żywności.
 \item[Modyfikacje planu] Modyfikacja może być bardzo kosztowna, jeśli będzie wiązać się z potrzebą przebudowy już zbudowanego obiektu ze względu na nagłe okoliczności.
 \item[Słaba jakość] Jest prawdopodobieństwo, że pracownicy mogą budować budowle o słabej jakości, nie stosować się do zaleceń i nie przykładać się do pracy. Infrastruktura musi starczyć na długo, dlatego musi być stworzona solidnie.
 \item[Za duże koszty tymczasowe] Do obsługi budowniczych będzie potrzebna infrastruktura tymczasowa, może się zdarzyć, że jej koszty będą za duże. Pieniądze wydane na obiekty tymczasowe pomogą zbudować, ale nie przybliżą do celu.
 \item[Uszkodzenia zbudowanych części] Czy to na skutek używania przez kolejne ekipy, albo przypadkiem. Powoduje opóźnienia i wymaga naprawy najlepiej przez tą samą grupę, która budowała obiekt.
 \item[Mieszany zespół] Zespoły nawet jeśli wcześniej wewnętrznie się znały, to nieraz muszą pracować z innymi zespołami z innych części świata. Są to ludzie o różnych rasach, językach i kulturach z którymi może być bardzo ciężko pracować.
 \item[Problemy interdyscyplinarności] Będą istnieć pojedyncze części projektu za duże do zbudowania przez jeden zespół. Łączenia między dziełami różnych zespołów są zawsze problematyczne ze względu na rzadszą komunikację między nimi.
 \item[Ataki terrorystyczne] Pomimo, że na teren będą wpuszczane tylko zaakceptowane i znane osoby, może się pojawić ryzyko sabotażu w końcowych fazach projektu.
 \item[Katastrofy naturalne] Islandia znana jest z aktywności wulkanicznej. Końcowy projekt ma być przygotowany na takie okoliczności, ale w czasie budowy może dojść do poważnych uszkodzeń.
 \item[Problemy prawne] Pomimo, że rząd Islandii może ustanowić w strefie specjalne prawa, nadal mogą istnieć prawdziwe, lub sztuczne wymagania co do innych państw ze względu na prawo międzynarodowe i podobne.
 \item[Brak zainteresowania sponsorów] Sponsorzy nie będą chcieli wnieść niczego do projektu oprócz pieniędzy. Nie zgłoszą zespołów, naukowców, ani nie będą sprawdzać postępu interesujących ich części.
\end{description}
Można przeanalizować najgroźniejsze z nich i wymyślić antidotum.

\subsection{Duża skala}
Aby dobrze zadziałać w dużej skali, należy mieć odpowiednio szczegółowo przedstawiony plan budowanej infrastruktury, a także plan jej budowy.
Niezmiernie ważna jest odpowiednio prosta komunikacja w zespole i zarządzie i automatyzacja jak największej ilości.
Pomoże też obecność małej ilości zespołów projektowych na raz. Zwiększy długość wykonania całości, ale znacząco poprawi przejrzystość projektu i wykrywanie błędów.

\subsection{Przekroczenie czasu}
Najczęściej spowodowane komplikacjami przy budowie takimi jak brak wystarczających kwalifikacji pracowników i prośbami o pomoc.
Rozwiązaniem będzie dokładniejsze sprawdzanie pracowników.

Może być wywołane przez opóźnienie niektórych zespołów. 
W takim wypadku dla operacji krytycznych należy zatrudnić najlepszych pracowników i dać im najwięcej kontroli i pomocy tak, aby prace zespołów nie miały opóźnień powodujących opóźnienia całego projektu.

\subsection{Przekroczenie budżetu}
Powodów przekroczenia budżetu jest nieskończenie wiele, ale na pewno zmniejszenie ryzyka zostanie osiągnięte poprzez sporządzenie odpowiedniego planu.
Należy w nim tak określić kolejność budowy, aby nie zapłacić za dużo za obiekty tymczasowe i za zmiany.

\subsection{Nieznana technologia}
Technologia jest nieznana. Pomimo wielu udanych testów nadal istnieje ryzyko, że coś może nie zadziałać tak, jak trzeba. Nikt nie ma doświadczenia z obchodzeniem się z nią.

Najważniejsza część projektu to kryształy do teleportacji. Bez nich działających nie ma sensu budować całej otoczki.
Jeśli nie uda się stworzyć ich systemu, cały projekt okaże się całkowitą porażką.

Dlatego ta jedna rzecz powinna być stworzona jako pierwsza na swoim miejscu, korzystając z tymczasowych rozwiązań transportowych i komunikacyjnych.
Całość musi działać, nawet jeśli nie będzie miała podstawowych udogodnień i zapewnienia bezpieczeństwa.
Należy wpierw przetestować działanie małych kryształków w laboratorium, aby nauczyć się ich obsługi i budowy systemów.

\subsection{Zła specyfikacja}
Aktywny i zaawansowany dialog ze sponsorami spowoduje, że zainteresują się oni projektem bardziej i dokładniej określą, czego oczekują od swojej części.
Dzięki temu możliwe będzie stworzenie lepszego planu.

\subsection{Zła kolejność budowy}
Zaczynanie budowy od zera wiąże się z odpowiednią kolejnością budowy systemów, jest duże ryzyko, że nie zawsze zapewni się wszystkie wymagania, co może spowolnić pracę innych. 
Najważniejsze jest poprawne zaplanowanie budowy, aby każdy miał to, czego do swojej części potrzebuje.

\subsection{Przeciągnięcie kosztów budowy modułu}
Aby nadmierny koszt jednej grupy nie wpłynął na innych, należy zostawić dla każdej grupy zapas kosztów.
Jeśli uda się grupie wypełnić zadanie w widełkach kosztu, jej zapas przechodzi na następnych.
To pozwoli częściowo zabezpieczyć się przed problemami po przekroczeniu budżetu.

\subsection{Przeciągnięcie czasu budowy modułu}
Należy zostawić zapas czasu dla każdej grupy tak, aby rozpoczęcie pracy kolejnej grupy nie było bezpośrednio po zakończeniu pracy poprzedniej.
W ten sposób nawet jeśli jakaś grupa opóźni się, to nie pociągnie ze sobą całego projektu, a najwyżej kilka grup.

\subsection{Brak kompetencji zespołów}
Należy sprowadzić takie zespoły projektowe, które budowały całe miasta od samego początku.
Dodatkowo trzeba eksperymentować z małym kryształami, aby nauczyć się jak budować teleportery z większych.

\subsection{Niepoprawna logistyka}
Organizacja tych wszystkich osób jest wyzwaniem, które bardzo trudno spełnić. Trzeba dowieść surowce, a także pożywienie dla wszystkich. 

Tak jak poprzednim razem, zmniejszenie jednoczesnej liczby pracujących pomoże zapanować nad wszystkimi i stosować tymczasowe rozwiązania.
Wydłuża to czas pracy, jednakże limit nie jest sztywno określony. Pomoże także stworzenie lepszego planu budowy.
 
\subsection{Modyfikacje planu}
Plan projektu nigdy nie będzie idealny, zwłaszcza jeśli ma obejmować swoim zasięgiem wszystko. Być może trzeba będzie modyfikować go w trakcie trwania projektu. 

Aby tego uniknąć, należy poświęcić na projektowanie więcej czasu. Należy myśleć w przód i dać do zweryfikowania wielu różnym osobom, aby ich inne myślenie pomogło wykryć słabe punkty

\subsection{Podsumowanie ryzyk}
Dla rozwiązywania dużej ilości powyższego ryzyka można zastosować te metody, które rozwiązują najwięcej z nich.

Aby uniknąć najwięcej ryzyka, należy szczególnie dokładnie stworzyć plan budowy, który będzie określał kiedy i gdzie kto pracuje, jak dowieźć sprzęt itp.
W ten sposób można zaplanować nawet reakcje na zdarzenia niezależne.

Budowa dokładnego planu infrastruktury pozwoli uniknąć sytuacji gdy jakieś elementy nie będą chciały ze sobą współpracować.
Uniknie to ponoszenia kosztów zmiany planu i przestojów.

Eksperymentowanie w kontrolowanych warunkach małymi wersjami kryształów pozwoli lepiej zaplanować przedsięwzięcie i uzyskać doświadczenie w zupełnie nieznanej dziedzinie.
Pozwoli to na wykrycie wczesnych błędów.

Rozciągnięcie projektu w czasie pozwoli na zmniejszenie ilości jednoczesnych grup budowlanych zwiększając tym samym kontrolę nad całością.
Pozwoli na lepszą organizację transportu kosztem dłuższego okresu zatrudnienia osób.