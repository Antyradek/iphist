\section{Analiza ryzyka}
Największe ryzyka, jakie grożą projektowi od najgroźniejszego, najbardziej prawdopodobnego:
\begin{description}
 \item[Duża skala] Skala projektu jest na tyle wielka, że istnieje bardzo wiele elementów powodujących ryzyko upadku projektu.
 \item[Przekroczenie czasu budowy] Projekt zostanie ukończony poprawnie, ale w za długim czasie.
 \item[Przekroczenie budżetu] Może być spowodowane bardzo wieloma czynnikami. Jest najpopularniejszym zagrożeniem większości projektów.
 \item[Nieznana technologia] Nowe miasta bardzo rzadko buduje się od podstaw. Również budowa stacji teleportujących nigdy wcześniej nie była przeprowadzana.
 \item[Zła specyfikacja] Może powodować, że trzeba będzie wprowadzać kosztowne zmiany, działać na obiektach tymczasowych, albo powodować, że struktury będą niepoprawnie działać.
 \item[Zła kolejność budowy] Nawet jeśli system jest poprawnie zaprojektowany, trzeba jeszcze stworzyć plan budowy. Bez niego budowniczy wyższych systemów mogą nie mieć wymaganych niższych i infrastruktury do życia. Będzie to generować dodatkowe koszty.
 \item[Przeciągnięcie kosztów budowy modułu] Jest bardzo prawdopodobne, że koszt wykonania pracy przez jedną ekipę może być większy, niż zaplanowano na skutek błędów i zdarzeń niezależnych.
 \item[Przeciągnięcie czasu budowy modułu] Każdy zespół buduje coś, na czym kolejni będą bazować. Przedłużenie budowy jednej części pociąga za sobą opóźnienia wszystkich innych.
 \item[Brak kompetencji zespołów] Nawet jeśli ogólna infrastruktura postanie poprawnie, to najtrudniej jest zbudować najbardziej zaawansowane systemy obsługujące kryształy. Nikt wcześniej nie próbował robić takich rzeczy i nikt nie ma doświadczenia.
 \item[Niepoprawna logistyka] Jeśli transport towarów i osób nie będzie poprawnie stworzony, to projekt z pewnością będzie powodował opóźnienia. Mogą powstać braki w zaopatrzeniu nie tylko materiałów budowlanych, ale także żywności.
 \item[Modyfikacje planu] Modyfikacja może być bardzo kosztowna, jeśli będzie wiązać się z potrzebą przebudowy już zbudowanego obiektu ze względu na nagłe okoliczności.
 \item[Słaba jakość] Jest prawdopodobieństwo, że pracownicy mogą budować budowle o słabej jakości, nie stosować się do zaleceń i nie przykładać się do pracy. Infrastruktura musi starczyć na długo, dlatego musi być stworzona solidnie.
 \item[Za duże koszty tymczasowe] Do obsługi budowniczych będzie potrzebna infrastruktura tymczasowa, może się zdarzyć, że jej koszty będą za duże. Pieniądze wydane na obiekty tymczasowe pomogą zbudować, ale nie przybliżą do celu.
 \item[Uszkodzenia zbudowanych części] Czy to na skutek używania przez kolejne ekipy, albo przypadkiem. Powoduje opóźnienia i wymaga naprawy najlepiej przez tą samą grupę, która budowała obiekt.
 \item[Mieszany zespół] Zespoły nawet jeśli wcześniej wewnętrznie się znały, to nieraz muszą pracować z innymi zespołami z innych części świata. Są to ludzie o różnych rasach, językach i kulturach z którymi może być bardzo ciężko pracować.
 \item[Problemy interdyscyplinarności] Będą istnieć pojedyncze części projektu za duże do zbudowania przez jeden zespół. Łączenia między dziełami różnych zespołów są zawsze problematyczne ze względu na rzadszą komunikację między nimi.
 \item[Ataki terrorystyczne] Pomimo, że na teren będą wpuszczane tylko zaakceptowane i znane osoby, może się pojawić ryzyko sabotażu w końcowych fazach projektu.
 \item[Katastrofy naturalne] Islandia znana jest z aktywności wulkanicznej. Końcowy projekt ma być przygotowany na takie okoliczności, ale w czasie budowy może dojść do poważnych uszkodzeń.
 \item[Problemy prawne] Pomimo, że rząd Islandii może ustanowić w strefie specjalne prawa, nadal mogą istnieć prawdziwe, lub sztuczne wymagania co do innych państw ze względu na prawo międzynarodowe i podobne.
 \item[Brak zainteresowania sponsorów] Sponsorzy nie będą chcieli wnieść niczego do projektu oprócz pieniędzy. Nie zgłoszą zespołów, naukowców, ani nie będą sprawdzać postępu interesujących ich części.
\end{description}
Można przeanalizować najgroźniejsze z nich i wymyślić antidotum.

\subsection{Duża skala}
Aby dobrze zadziałać w dużej skali, należy mieć odpowiednio szczegółowo przedstawiony plan budowanej infrastruktury, a także plan jej budowy.
Niezmiernie ważna jest odpowiednio prosta komunikacja w zespole i zarządzie i automatyzacja jak największej ilości.
Pomoże też obecność małej ilości zespołów projektowych na raz. Zwiększy długość wykonania całości, ale znacząco poprawi przejrzystość projektu i wykrywanie błędów.

\subsection{Przekroczenie budżetu}
Powodów przekroczenia budżetu jest nieskończenie wiele, ale na pewno zmniejszenie ryzyka zostanie osiągnięte poprzez sporządzenie odpowiedniego planu 

\subsection{Nieznana technologia}
Technologia jest nieznana. Pomimo wielu udanych testów nadal istnieje ryzyko, że coś może nie zadziałać tak, jak trzeba. Nikt nie ma doświadczenia z obchodzeniem się z nią.

Najważniejsza część projektu to kryształy do teleportacji. Bez ich działających nie ma sensu budować całej otoczki.
Jeśli nie uda się stworzyć ich systemu, cały projekt okaże się całkowitą porażką.

Dlatego ta jedna rzecz powinna być stworzona jako pierwsza na swoim miejscu, korzystając z tymczasowych rozwiązań transportowych i komunikacyjnych.
Całość musi działać, nawet jeśli nie będzie miała podstawowych udogodnień i zapewnienia bezpieczeństwa.

Na szczęście zawsze można przerobić nieudany projekt na nowe miasto na pustkowiu.
Jeśli wprowadzić liberalne prawo, nawet zadeklarować jako osobne państewko, to miasto mogłoby się rozwijać pomimo, że pierwotnie miało być przeznaczone do czegoś innego.

\subsection{Mieszany zespół}
Zespół składa się z wielu osób z całego świata. Pomimo dużej kontroli i restrykcyjnych warunków, niektórzy mogą nie mieć dość profesjonalizmu, aby dobrze pracować. Powstają też bariery kulturowe i językowe. 

Należy umieszczać podobne osoby w jednym zespole, aby uniknąć wewnętrznych tarć. Mimo że mieszany zespół teoretycznie może być szerzej wykształcony, to wewnętrzny spokój mentalny jest ważniejszy.
Także komunikacja między zespołami powinna być prosta. Najlepiej bez pośredników i tłumaczy. Lider zespołu powinien móc dogadać się ze wszystkimi innymi.

\subsection{Zła kolejność budowy}
Zaczynanie budowy od zera wiąże się z odpowiednią kolejnością budowy systemów, jest duże ryzyko, że nie zawsze zapewni się wszystkie wymagania, co może spowolnić pracę innych. 

Lepiej jest wydłużyć czas budowy używając mniej zespołów na raz, a zapewnić większą kontrolę. Dzięki temu czasem można zapewnić coś tymczasowo nie wstrzymując rozwoju.
Jednakże takie akcje niepotrzebnie zużywają budżet, dlatego planowanie jest ważniejsze.

\subsection{Organizacja}
Organizacja tych wszystkich osób jest wyzwaniem, które bardzo trudno spełnić. Trzeba dowieść surowce, a także pożywienie dla wszystkich. 

Tak, jak poprzednim razem, zmniejszenie jednoczesnej liczby pracujących pomoże zapanować nad wszystkimi i stosować tymczasowe rozwiązania.
Wydłuża to czas pracy, jednakże limit nie jest sztywno określony.
 
\subsection{Modyfikacje planu}
Plan projektu nigdy nie będzie idealny, zwłaszcza jeśli ma obejmować swoim zasięgiem wszystko. Być może trzeba będzie modyfikować go w locie. 

Aby uniknąć tego, należy poświęcić na projektowanie więcej czasu. Należy myśleć w przód i dać do zweryfikowania wielu różnym osobom, aby ich inne myślenie pomogło wykryć słabe punkty

\subsection{Podsumowanie ryzyk}
Dla rozwiązywania dużej ilości powyższego ryzyka można zastosować łączone metody.
% Dokładny plan infrastruktury
% Dokładny plan budowy
% Dobra komunikacja
% Rozciągnięcie projektu i zwiększenie dokładoności


