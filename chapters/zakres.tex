\section{Zakres} 
\subsection{Cele biznesowe}
Głównym celem jest umożliwienie przenoszenia obiektów i ludzi na inne planety.
Pozwala to na:
\begin{description}
\item[Handel] Uożliwienie sprzedaży naszych zasobów naturalnych w zamian za inne dobra oraz kupowanie kosmicznych wyrobów.
\item[Poczta] Rozszerzenie poczty lotniczej lub morskiej na międzyplanetarną.
\item[Turyzm] Przewidywane liczne grono chętnych do wizyty naszej planety mieszkańców innych planet (a także Ziemian pragnących zwiedzić obce światy), umożliwiające rozwój gospodarczy wielu państw.
\item[Nauka] Ułatwione badania dzięki nawiązaniu współpracy naukowców. Również szybki transport w inne rejony galaktyki, pozwalający ludzkości na skok technologiczny w eksploracji kosmosu.
\item[Siła robocza] Delegowanie produkcji na inne planety alternatywą dla taniej siły roboczej na Ziemi.
\item[Podatki] Znaczny przypływ gotówki do kasy Islandii oraz inwestorów wskutek pobierania cła oraz opłat za używanie teleporterów.
\end{description}

\subsection{Aspekt użytkowy}
Standardowy użytkownik będzie mógł zakupić bilet, obejmujący również prom na Islandię lub przejazd planowanym Mostem Atlantyckim z Europy oraz pobyt w hotelu.
Osobne miasteczko będzie służyło jako bufor dla odwiedzających.
Będą się tam mieścić hotele, parkingi, podstawowe sklepy i miejsca do przeładunku, a także biura.

Klient wraz ze swoim samochodem lub innym pojazdem zostanie skierowany w kolejności do jednego z wielu wejść do nadajnika.
W inne miejsca kierowane będą transporty dóbr z kontenerami.
Tam każdy obiekt (w tym człowiek lub przybysz z Kosmosu) będzie odpowiednio prześwietlany w poszukiwaniu ładunków wybuchowych mogących zniszczyć teleporter.
Nastąpi także szczegółowa kontrola biletowa i paszportowa.

Wszystkie drogi będą się spotykać pod wielkim kryształem, który będzie zamieniał materię pod sobą na promień przesyłany przez antenę do odbiornika.
Każde uruchomienie kryształu wyśle materię do innego odbiornika, toteż ważne jest, aby pojazdy udające się w to samo miejsce znalazły się pod kryształem razem.
Do tego wymagany jest system kolejek kierujący pojazdy grupami do odpowiednich tuneli.
Gdy nadejdzie odpowiednia chwila, oczekujący na końcu tunelu kierowani będą na środek, a kryształ uruchomi się z zaprogramowanym odbiornikiem.

W osobnym budynku znajdzie się odbiornik.
Nie będzie on duży, gdyż wymaga małej infrastruktury.
Podobnie do nadajnika, materia odebrana promieniem pojawi się pod kryształem i będzie zmuszona jak najszybciej opuścić odbiornik, aby umożliwić korzystanie z odbiornika kolejnym użytkownikom.
Cały ruch kierowany będzie do wielkiego, zadaszonego i szczelnego bufora, na końcu którego znajdować się będzie wiele równoległych stanowisk.
Na stanowiskach pojazdy będą przewietlane pod kątem niebezpiecznych dla życia na Ziemi substancji, odbędzie się również kontrola dokumentów.
W przypadku nieudanej weryfikacji pojazd kierowany będzie specjalną drogą do nadajnika powrotnego.

\subsection{Aspekt techniczny}
Otrzymaliśmy od Handthów dwa wielkie kryształy, jeden pozwalał na zamianę materii w energię, a drugi odwrotnie. 
Stanowić one będą rdzeń całego urządzenia. 
Dołączone do niego są instrukcje obsługi, a także plany budowy.
Postanowiono na ich podstawie zbudować ziemskie kopie, a oryginały trzymać jako zapasowe, aby mieć pełną kontrolę nad ich działaniem.

Interakcja polega na wysyłaniu sygnałów elektrycznych, umożliwiających dokładne sterowanie ich pracą.
Kryształ wysyła promień w kierunku swojego wierzchołka, który zamienia znalezioną materię w pakiety energii.
Drugą stroną wysyła sygnał do anteny kierującej do przekaźnika w kosmosie, który to kieruje do następnego itd. aż do odbiornika.

Postanowiono oprzeć działanie na nowych technologiach.
Stwierdzono, że cały kod systemu powinien być otwarty.
Pozwoli to na łatwą kontrolę błędów przez osoby trzecie i zwiększy jego jakość poprzez presję programistów.
Dodatkowo elektronika także powinna być otwartoźródłowa, aby mieć całkowitą pewność o jej prawidłowym działaniu bez ukrytych tylnych drzwi.

Infrastruktura fizyczna powinna być duża i zbudowana jak bunkier, aby wytrzymać próbę czasu oraz katastrofy naturalne.
Przekłada się funkcjonalność nad wygląd i wygodę.
Jednocześnie wszystko powinno być proste i oczywiste w obsłudze.

Minimalizuje się ilość urządzeń elektronicznych i mechanicznych, aby uniknąć dużej ilości awarii.
Całkowicie pomija się mało znaczące aspekty, jak wpływ infrastruktury na środowisko.

Bardzo duże znaczenie ma sprawny transport zarówno dla ludzi, jak i dla towarów.
Dobrym pomysłem jest umieszczenie podziemnych obrotowych pierścieni wokół centrum, które będą działać jak ruchomy chodnik. Nie trzeba będzie czekać na wagony.
Miasto będzie się znajdować na tym pierścieniu wokół budowli nadajnika i odbiornika.

Do transportu towarów nada się transport naziemny, oraz kolejki jednoszynowe uzupełniające pierścienie.

\subsection{Cel}
Celem jest stworzenie działającego systemu łączącego Ziemię z najbliższym przekaźnikiem tak, aby zachować pełną kontrolę nad jego działaniem.

Cały projekt powstaje z inicjatywy obywateli i nikt nie może sprawować nad tym władzy absolutnej. Zarząd jest demokratyczny i składa się z polityków Islandii, największych inwestorów i obywateli.

Infrastruktura musi być samowystarczalna, energia elektryczna produkowana jest nieopodal, a miasto posiada fabryki jedzenia oraz domy dla pracowników. Całość może funkcjonować jako oblężona forteca.

Należy przede wszystkim zadbać o bezpieczeństwo na odbiorniku, aby wykrywać i blokować wszystkie niebezpieczne kosmiczne substancje, które mogą nieść zagrożenie dla Ziemi w skali globalnej.
Bezpieczeństwo samej infrastruktury jest drugorzędne, należy wykrywać i przeciwdziałać atakom na nią samą.

Konstrukcja musi być wytrzymała, aby mogła wystarczyć na długo, oraz wytrzymywać ewentualne katastrofy naturalne.

Piękno i wygoda nie powinny zaćmić funkcjonalności. Elegancja nie jest potrzebna.

Sprawa legalności przewożonych produktów nie powinna być brana pod uwagę przez infrastrukturę, a przez państwa przyjmujące ruch z Islandii.

Jakiekolwiek ideologie i odczucia innych nie mogą uniemożliwić dodania funkcjonalności do systemu. 

Sukcesywna budowa jest ważniejsza, niż dbanie o środowisko. Należy wykorzystać wszystko, aby stworzyć nowoczesny system.

%pewnie coś jeszcze
