\section{Zakres} 
\subsection{Cele biznesowe}
Głównym celem jest umożliwienie przenoszenia obiektów i ludzi na inne planety.
To pozwala na:
\begin{description}
\item[Handel] Możliwe będzie sprzedawanie naszych zasobów naturalnych w zamian za inne dobra, oraz kupowanie kosmicznych wyrobów.
\item[Poczta] Rozszerzenie poczty lotniczej, lub morskiej na międzyplanetarną.
\item[Turyzm] Dla wielu krajów turyści to podstawowy dochód. Na pewno będzie bardzo wielu chętnych do wizyty naszej planety, a także wielu ziemian będzie chciało zwiedzić obce światy.
\item[Nauka] Ułatwione badania dzięki nawiązaniu współpracy naukowców. Również szybki transport w inne rejony galaktyki pozwoli ludzkości na skok technologiczny w eksploracji kosmosu.
\item[Siła robocza] Tania siła robocza jest głównym atutem niektórych krajów. Lepsze od tego może być tylko delegowanie produkcji do innych planet.
\item[Podatki] Zbieranie cła oraz normalnych opłat za używanie teleporterów spowoduje spory przypływ gotówki do kasy Islandii, oraz inwestorów.
\end{description}

\subsection{Aspekt użytkowy}
Standardowy użytkownik będzie mógł zakupić bilet obejmujący również prom na Islandię, lub przejazd planowanym Mostem Atlantyckim z Europy, oraz pobyt w hotelu.
Osobne miasteczko będzie służyło jako bufor dla odwiedzających.
Będą się tam mieścić hotele, parkingi, podstawowe sklepy i miejsca do przeładunku, a także biura.

Klient wraz ze swoim samochodem, lub pojazdem komunikacji zostanie skierowany w kolejności do jednego z wielu wejść do nadajnika.
W inne miejsca kierowane będą transporty dóbr z kontenerami.
Tam każda rzecz i człowiek będzie odpowiednio prześwietlana w poszukiwaniu ładunków wybuchowych mogących zniszczyć teleporter.
Nastąpi także szczegółowa kontrola biletowa i paszportowa.

Wszystkie drogi będą się spotykać pod wielkim kryształem, który będzie zamieniał materię pod sobą na promień przesyłany przez antenę do odbiornika.
Każde uruchomienie kryształu wyśle materię do innego odbiornika, toteż ważne jest, aby pojazdy udające się w to samo miejsce znalazły się pod kryształem razem.
Do tego wymagany jest system kolejek kierujący pojazdy grupami do odpowiednich tuneli.
Gdy nadejdzie odpowiednia chwila, oczekujący na końcu tunelu kierowani są na środek, a kryształ uruchamia się z zaprogramowanym odbiornikiem.

W osobnym budynku znajduje się odbiornik.
Nie jest on duży, gdyż wymaga małej infrastruktury.
Podobnie do nadajnika, materia odebrana promieniem pojawia się pod kryształem i musi jak najszybciej opuścić miejsce, aby pozwolić kolejnym na powrót.
Cały ruch kierowany jest do wielkiego, zadaszonego i szczelnego bufora na końcu którego znajduje się wiele równoległych stanowisk.
Prześwietlają one pojazdy pod kątem niebezpiecznych dla życia na Ziemi substancji, oraz sprawdzają dokumenty.
Jeśli weryfikacja się nie udała, pojazd kierowany jest specjalną drogą do nadajnika.

\subsection{Aspekt techniczny}
Otrzymaliśmy od Handthów dwa wielkie kryształy, jeden pozwalał na zamianę materii w energię, a drugi odwrotnie. 
Stanowić one będą rdzeń całego urządzenia. 
Dołączone do niego są instrukcje obsługi, a także plany budowy.
Postanowiono na ich podstawie zbudować ziemskie kopie, a oryginały trzymać jako zapasowe, aby mieć pełną kontrolę nad ich działaniem.

Interakcja polega na wysyłaniu sygnałów elektrycznych, aby sterować ich dokładną pracą.
Kryształ wysyła promień w kierunku swojego wierzchołka, który zamienia znalezioną materię w pakiety energii.
Drugą stroną wysyła sygnał do anteny kierującej do przekaźnika w kosmosie, który to kieruje do następnego itd. aż do odbiornika.

Postanowiono oprzeć działanie na nowych technologiach.
Wymogiem sponsorów jest to, aby cały kod był otwarty.
Pozwoli to na łatwą kontrolę błędów przez osoby trzecie.
Dodatkowo elektronika także powinna być otwartoźródłowa, aby mieć całkowitą pewność o jej prawidłowym działaniu bez ukrytych tylnych drzwi.

Infrastruktura fizyczna powinna być duża i zbudowana jak bunkier, aby wytrzymać próbę czasu, oraz katastrofy naturalne.
Przekłada się funkcjonalność nad wygląd i wygodę.
Jednocześnie wszystko powinno być proste i oczywiste w obsłudze.

Minimalizuje się ilość urządzeń elektronicznych i mechanicznych, aby uniknąć dużej ilości awarii.
Całkowicie pomija się mało znaczące aspekty, jak wpływ infrastruktury na środowisko.

Bardzo duże znaczenie ma sprawny transport zarówno dla ludzi, jak i dla towarów.
Dobrym pomysłem jest umieszczenie podziemnych obrotowych pierścieni wokół centrum, które będą działać jak ruchomy chodnik. Nie trzeba będzie czekać na wagony.
Miasto będzie się znajdować na tym pierścieniu wokół budowli nadajnika i odbiornika.

Do transportu towarów nada się transport naziemny, oraz kolejki jednoszynowe uzupełniające pierścienie.

\subsection{Cel}
Stworzenie działającego systemu łączącego Ziemię z najbliższym przekaźnikiem tak, aby zachować pełną kontrolę nad jego działaniem.

Wszystko to powstało z inicjatywy obywateli i nikt nie może sprawować nad tym władzy absolutnej. Zarząd jest demokratyczny i składa się z polityków Islandii, największych inwestorów i obywateli.

Infrastruktura musi być samowystarczalna, energia elektryczna produkowana jest nieopodal, a miasto posiada fabryki jedzenia, oraz domy dla pracowników. Całość może funkcjonować jako oblężona forteca.

Należy przede wszystkim zadbać o bezpieczeństwo na odbiorniku, aby wykrywać blokować wszystkie niebezpieczne kosmiczne substancje, które mogą nieść zagrożenie dla Ziemi w skali globalnej.
Bezpieczeństwo samej infrastruktury jest drugorzędne, należy wykrywać i przeciwdziałać atakom na nią samą.

Konstrukcja musi być wytrzymała, aby mogła wystarczyć na długo, oraz wytrzymywać ewentualne katastrofy naturalne.

Piękno i wygoda nie powinny zaćmić funkcjonalności. Elegancja nie jest potrzebna.

Sprawa legalności przewożonych produktów nie powinna być brana pod uwagę przez infrastrukturę, a przez państwa przyjmujące ruch z Islandii.

Jakiekolwiek ideologie i odczucia innych nie mogą uniemożliwić dodania funkcjonalności do systemu. 

Sukcesywna budowa jest ważniejsza, niż dbanie o środowisko. Należy wykorzystać wszystko, aby stworzyć nowoczesny system.

%pewnie coś jeszcze
